\documentclass{article}
\usepackage{geometry}
\geometry{a4paper}
\usepackage{listings}
\usepackage{graphicx} 
\usepackage[ngerman]{babel}
\usepackage[utf8]{inputenc}
\usepackage[T1]{fontenc}    
\usepackage{amsmath}
\usepackage{float}
\usepackage{url}

\title{
	Bericht: Simulation von Bewegungen \\
	\large Raumsonde mit steuerbarem Antrieb: Flug zu einem Planeten}
\author{Amin Nacer Cherif, Simon Wohler, Joel André}

\begin{document}

\maketitle

\section{Auftrag}
In unserem Projekt geht es darum, dass man im Sonnensystem von der Erde aus eine Raumsonde gezielt auf, bzw in die Umlaufbahn eines anderen Planeten(z.B. Saturn) schiessen kann. Dabei soll ein swing-by Manöver um einen dazwischenliegenden Planeten(z.B. Jupiter) vollführt werden.

	\subsection{Persönliche Ziele}
		\begin{itemize} 
			\item Simulation des Sonnensystems
			\item Zwei verschiedene Modi
			\begin{itemize} 
				\item Manuell steuerbare Raumsonde
				\item Raumsonde fliegt festgelegtes swing-by Manöver
			\end{itemize}
		\end{itemize}

\section{Motivation}
Gewählt wurde das Thema, weil es eine Herausforderung darstellt und gleich mehrere zu lösende Probleme auf einmal enthält. Zudem sind wir eine Gruppe aus drei Personen, wodurch es möglich ist, eine Problemstellung mit grösserem Umfang zu bearbeiten. Alle Gruppenmitglieder empfinden das Thema als spannend und interessant zum Bearbeiten.

\section{Vorgehen und Reflexion}
Als erster Schritt wurde festgehalten, welche Teilschritte nötig sind. So wusste immer jeder, was es noch zu erledigen gab und konnte sich nach einem abgeschlossenen Schritt sofort dem nächsten widmen. 
Es wurden dazu unter anderem \textit{Github Issue's} verwendet. 

Zu Beginn wurde das \textit{Runge-Kutta} Verfahren programmiert. Dabei hatten wir anfängliche Verständnis Schwierigkeiten bezüglich was die DGL sei. Schnell schien das Verfahren aber zu funktionieren, da sich die Planeten für eine kurze Zeit korrekt bewegten. Allerdings nahmen sie immer mehr Abstand zur Sonne, was eindeutig nicht sein sollte.  Der Fehler wurde nicht gleich gefunden, doch konnte festgestellt werden, dass er irgendwo im Programmcode steckt, denn die Anfangsbedingungen wurden alle überprüft und mit den Angaben der NASA verglichen. Herausgestellt hat sich dann, dass es lediglich ein Flüchtigkeitsfehler war und in der Differenzialgleichung auf alte Positionen zugegriffen wurde.
Gleichzeitig widmete sich jemand anderes dem Programmieren eines Sonnensystems mit \textit{Pygame}. Weiter wurde nach Anfangsbedingungen für das Sonnensystem gesucht. Als Schwierigkeit stellte sich dabei die Berechnung der Anfangsgeschwindigkeiten der Planeten heraus. Durch Herr Kambors physikalische Unterstützung konnten diese jedoch sehr genau berechnet werden. 

Als nächstes wurde damit begonnen ein Menu zu programmieren und das fertiggestellte \textit{Runge-Kutta} Verfahren in das main-Programm zu implementieren.  Zudem wurde das Menu implementiert, in welchem man auswählen kann, ob man die Rakete selbst steuert oder ob sie simuliert werden soll. Als weitere Funktion kann man im Menu einige wichtige Parameter festsetzen. Dies funktionierte aber noch nicht, denn trotz eingegebenen Daten ändert sich gar nichts an der Simulation. Dann wurde eine Zoom-Funktion geschrieben, mit welcher man direkt auf einen Planeten zoomen kann. Parallell dazu hat man eine Funktion geschrieben, damit man manuell bestimmen kann, wie schnell die Zeit in der Simulation vergeht. 
Als letzter Schritt wurde der ganz Programmcode nochmals sorgfältig überarbeitet und kommentiert

\section{Produkt (aktueller Stand)}
\begin{figure}[hbt]
	\centering
		\includegraphics[width=0.8\textwidth]{game.png}
	\caption{Simulation}
	\label{img:simulation}
\end{figure}
\subsection{Simulation des Sonnensystems}
Aktuell, kann das Programm erfolgreich das Sonnensystem mit ausgewählten Planeten simulieren. Simuliert werden Erde, Mars, Jupiter und Saturn. Die Anfangsbedingungen stimmen so gut wie möglich mit dem Sonnensystem überein.
\subsubsection{Implementation des \textit{Runge-Kutta }Verfahrens}
Das \textit{Runge-Kutta} Verfahren wird lediglich einmal durchgeführt für alle Planeten, bzw. alle Planeten gehen gleichzeitig durch das Verfahren. Das liegt daran, dass beim Schritt nach vorne die neue Position der anderen Planeten ebenfalls berücksichtigt werden muss.

In der DGL werden immer zuerst alle Beschleunigungen addiert, die auf einen Körper wirken.
\subsection{Interface}
\subsubsection{Menu}
Ein Menu wurde ebenfalls implementiert. Allerdings ist es noch voller versteckter Fehler. Es gibt momentan zwar schon viele Knöpfe, aber viele sind noch ohne Funktion. Später soll man dan zwischen den Modi wechseln können und Konstanten ändern.
\subsubsection{Zoom}
Ein Zoom war wichtig für das Spaceshuttle. Ohne Zoom wäre es unmöglich die Sonde in Erdnähe präzis zu steuern. Wenn man auf Körper drückt mit der Maustaste kann man das Zentrum ändern.

\textit{p} vergrössert und \textit{m} verkleinert
\subsubsection{Geschwindigkeit}
Der Geschwindigkeitsfaktor wird realistisch angezeigt. Das Spulen nach vorne war ebenfalls nötig um bei langsamer Geschwindigkeit das Shuttle zu steuern und dann bei schneller Geschwindigkeit Resultate sehen.

\textit{arrow up} beschleunigt und \textit{arrow down} verlangsamt.

\subsubsection{Spur}
Eine Spur wurde zur Übersichtlichkeit hinzugefügt. Die Länge der Spur lässt sich als Zeit [s] einstellen. Momentan hat die Spur bei allen Planeten, die Länge ihrer Umlaufzeit. Dass die Spur sich schön schliesst, beweist dass die Umlaufzeit der Planeten mit der echten Umlaufzeit übereinstimmt.


\section{Reflexion und Ausblick}
Der nächste Schritt in unserer Arbeit wird sein, die Raumsonde zu simulieren. Dazu wird sie der selben Klasse zugewiesen, wie die anderen Planeten auch. D.h. sie wird wie ein mini Planet behandelt. Die Raumsonde ist mit einem \textit{Vulcain 2} Triebwerk ausgestattet und sollte dann auch dementsprechend beschleunigen können. Zudem sollte die Raumsonde gesteuert werden können, indem man den Winkel, in welchem die Rakete fliegt ändern kann. Vorgesehen ist, dass als Anfangswert die Raumsonde direkt an der Erde fixiert wird. 

Um beim steuern abzuschätzen zu können in welche Richtig die Raumsonde fliegt, werden wir versuchen eine Spur vor der Raumsonde anzeigen zu lassen. Diese Spur wird zeigen, wo die Raumsonde hinfliegen würde, falls nicht beschleunigt wird.

Das Wunschziel unserer Gruppe ist es, dann noch ein kleines Programm schreiben, welches direkt die optimale Route zu einem anderen Planeten berechnet und dies, wenn möglich, auch mit Hilfe eines swing-by Manövers.

\end{document}